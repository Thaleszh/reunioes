\documentclass{ata-calico}

\begin{document}

\maketitle

\pauta{Formação da comissão organizadora}

João Paulo conta como Vanessa, ele, Cauê, Mikael, Lucas, Thales, Helena e Inã
demonstraram interesse. Vanessa ainda adiciona Marcelo e Victor, de sistemas.
Lucas diz que devemos reunir essas 10 pessoas e que mais pessoas não são
necessárias, uma vez que muita gente se mostrou ruim durante o PET\@. Luís
comenta como seria bom ter 5 pessoas de sistemas e 5 pessoas de computação, o
que é concordado. Lucas diz que durante a edição passada foi separado entre uma
comissão que sabe da organização e outras que só sabem os trabalhos que lhes
foram atribuídas, Luis diz que o que faremos agora e Lucas responde que 6 seria
ideal por refletir as 6 frontes como a organização foi feita semestre passado,
Luis responde que teremos uma comissão e não necessariamente isso será feito
como nas edições anteriores e 10 parece um bom número, uma vez que a comissão
gerenciaria e não necessariamente estarão em um grupo de função específica.
Lucas questiona o fato de membros da comissão não estarem em uma ``fronte''.
Vanessa comenta como acredita que essas pessoas são para decidir o que será
feito e como/quem deverá fazer, Luis diz que é isso que quer dizer. Lucas
menciona como só temos 2 mulheres e seria legal ter mais, Vanessa diz que
conhece algumas meninas de sistemas que poderiam ter interesse em participar, e
que conseguem pelo menos uma. Mikael ofereceu-se para se retirar uma vez que
existe um excesso de voluntários para sistemas. Cauê diz estar confuso e Lucas
diz que a comissão é quem toma decisões, Cauê diz que não acha necessário essa
segregação e Lucas responde que é melhor um grupo assumir para si a
responsabilidade, e Luis adiciona que essa comissão pode fazer decisões em
reuniões abertas. Cauê diz que não entende uma divisão ainda uma vez que
queremos apenas elencar pessoas com interesse ajudar e não há motivo para
segregação. João Pauo dá o exemplo de que se você precisa fazer algo feito é
mais fáci falar com alguém dessa comissão, Vanessa apoia e ainda diz ser
improtante ter o mesmo número de pessoas para que os dois cursos tenham a mesma
quantidade de responsabilidade. João Gabriel comenta como não se importa em
sair uma vez que não possui experiência e que pode ajudar a SECCOM de outras
maneiras. Lucas diz que estamos supondo formas de organização e que ainda não
sabemos como isso será feito e que discussão de como coisas serão feitas são
meio irrelevantes. Caue pede por que tiraremos uma pessoa que quer ajudar, Luis
responde que ela poderá ajudar apenas não estar na comissão, Cauê tem a ideia
de que todos podem fazer o evento de maneira completamente horizontal uma vez
que não haverão muitos interessados, Lucas responde como é melhor deixar
pessoas responsáveis, Mikael concorda, Luis diz que isso deixa uma brecha
aberta. Cauê ainda discorda já que todos poderiam conversar e delegar as
responsabilidades em grupo, Vanessa diz que é preciso elencar pessoas uma vez
que precisaremos de responsáveis para redigir documentos que possam ser
necessários na instituição. Lucas pede se iremos tirar alguém da comissão para
colocar uma mulher, Lucas acha desnecessário, mesmo que representatividade seja
importante mas tirar alguém apenas para isso. Cauê diz que acima disso a
representatividade mais importante é a dos dois cursos.

Pessoas da comissão serão notificadas e aguardaremos futuras instruções. Serão
repassadas para elas documentos de como eventos passados foram feitos para que
possam ter uma base.

\pauta{Reserva de espaço físico} João Paulo diz que não fizemos reserva da
reitoria e que deve ser feita um ano antes. Lucas diz que existe um edital para
reservas e que ele já o fez, Cauê disse que leu o manual da SECCOM disse que a
entidade responsável pelo edital não existe mais, Lucas responde que ano
passado ele teve contato com eles e Cauê diz que agora regras de reserva estão
em um departamento e que lá diz que é algo novo e não menciona editais. Lucas
diz que a reitoria é um ótimo espaço, bem localizado, coisas no saguão chamam
atenção, mas ao mesmo tempo é uma atenção que pode chamar pessoas que não são o
público alvo. Lucas continua falando que ano passado foi colocado no Hall do
CTC e que as empresas disseram que a movimentação foi tão forte quanto a da
Work Week e portanto parece um ótimo lugar, uma vez que quase toodas as pessoas
que transitam no local tem alguma coisa a ver com o evento, porém se ela for
usado a reitoria se torna um lugar com uma distância inconveniente. Lucas ainda
diz que os auditórios até um mês antes pessoas do departamento tem preferência,
Cauê diz que no EPS apenas permite reservas duas semanas antes. João Gabriel
adiciona que podemos reservar em nome do professor elencado como responsável,
Lucas esclarece que sim porém apenas nos auditórios do departamento eles teriam
preferência. O EPS e a reitoria possuem um bom espaço e fácil de fechar na hora
de coffee breaks, o hall do EEL possui bancadas e também é possível encontrar
uma sala próxima.  Lucas ainda adiciona que o auditório da reitoria é grande de
mais, o que sempre dá a impressão de que está vazio, Cauê diz que teremos esse
mesmo problema no EFI e que EPS e EEL parecem mais adequados e CTC muito
pequeno. Tiz concorda com Cauê. Vanessa diz que pessoal teve um aproveitamento
pequeno pois terminou bastante cedo. Lucas diz que com relação à feira
estávamos usando muito da movimentação do local e a intenção foi não deixar até
muito tarde quando a movimentação era pequena, e disse como inclusive empesas
pararam de ir no decorrer do evento quando o fluxo de pessoas foi diminuindo.
Lucas adiciona que é preciso verificar laboratórios caso sejam dados
minicursos.

\pauta{Sugestões de professores organizadores} Cauê diz que Plentz sugeriu
Maicon e Jerusa, que ele lembre. Lucas falou que Jerusa é muito ocupada e Luis
e Caue diz que isso seria bom para que não imponham coisas na semana acadêmica.
Vanessa comenta Delucca. Cauê diz o Jean Hauck e diz que seria importante ser
alguém que a Plentz aprove uma vez que ela possui o projeto atual. Lucas
comenta como em versões anteriores o projeto foi feito pela FEESC
principalmente pela facilidade ao lidar com pagamentos de empresas, porém como
desvantagem tudo que quiser ser comprado deverá ser requisitado por eles e eles
sempre irão comprar a opção mais em conta, portanto tudo que for pedido deve
ser muito bem especificado para que venha corretamente. Outro problema é que o
projeto na FEESC precisa ser fechado, ele também precisa de um professor
associado e um projeto não pode ser extendido por dois anos, depois disso pode
ser criada uma conta na FEESC que pode ser prorrogada ad infinitum. Lucas diz
que outra possibilidade é ter um CNPJ e ir atrás do pessoal, Cauê pede como
está situação de CNPJ do CASIN e Marcelo diz que não é possível utilizá-lo.
Luis diz que a vantagem de usar o CNPJ é não ligar a semana acadêmica com um
professor e que seria interessante ter uma comissão permanente que consiga
sempre perpetuar a semana acadêmica, uma entidade independente feita através de
indicações como é feito na amnésia. João Paulo diz que se preocupa em casos de
"CA decadente" e Luis comenta que um evento não tem motivo de existir se
ninguém tem interesse nele, João Paulo responde que é questão de quem deveria
ter interesse e que não tem. Luis continua dizendo que se chegar um ponto em
que ninguém queira tocar, ele não precisa ser tocado, mas isso não é relevante
para a pauta atual uma vez que não conseguiremos um CNPJ em tempo hábil. Cauê
comenta como Martina interferiria muito pouco e Luis e Lucas mencionam que é
isso que estamos procurando. Lucas pede se Martina colocaria mais pressão nos
professores para apoiar o evento, Luis diz que independente disso isso é um
movimento que deve partir dos alunos e que damos nosso jeito. Vanessa diz que
em questão de liberar alunos deve ser algo muito bem alinhado com a
coordenação, Luis adiciona que se os alunos realmente tiverem interesse
pressionariam alunos. Cauê disse que isso não aconteceria e mesmo com interesse
eles parmaneceriam inertes, Vanessa diz que calouros teriam dificuldade de se
impor. Luis diz que teria a força da turma inteira, João Paulo e Cauê dizem que
40 alunos pode ser pior do que um. Luis diz que isso pode ser encabeçado pelos
próprios CAs e não necessariamente pela turma. Marcelo diz que depender da
coordenação é algo complicado uma vez que suas ações podem ser limitadas. Lucas
comenta como o máximo que o coordenador do departamento fez foi enviar um
e-mail e disse que não pode obrigá-los a nada. Luis diz que a mudança não deve
vir de baixo para cima e que o interesse deve primeiramente partir dos alunos e
como o quórum foi baixo os professores também não irão se importar. João Paulo
diz que também é dependente de coisas como trabalho. Luis comenta como outros
cursos conseguem mudar coisas com movimento estudantil e não há motivo para o
CTC não conseguir mas eles não tem engajamento. Cauê diz que semana que vem a
comissão deve se reunir para conseguir decidir a data, para fazer reservas e
adicionar ao calendário do curso. Mikael diz que não importa uma vez que os
professores já fizeram seus cronogramas. Lucas responde como o cronograma é
semestral. Mikael sugere Lúcia.

Professores viáveis são Maicon, Jerusa, Delucca, Jean Hauck, Jean Martina,
Lúcia.

\pauta{Fechamento}

É concordado que demais pautas deveriam ser discutidas pela comissão, Luis diz
como o mais interessante atualmente seria marcar uma reunião. A reunião foi
marcada para quarta-feira. João Paulo sugere issues no github para delegar
tarefas, Marcelo e Vanessa demonstram apoiar.

\presentes{%
    Cauê Baasch,
    João Gabriel Trombeta,
    João Paulo T\@.,
    Lucas Sousa,
    Luis Oswaldo Ganoza
    Marcelo Brosowicz,
    Mikael Saraiva,
    Vanessa Cunha
}

\end{document}
